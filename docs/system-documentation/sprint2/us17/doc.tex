%! Author = Bruno Silva 1221514
%! Date = 15/05/2025

\documentclass{article}
\usepackage{amsmath}
\usepackage{graphicx}
\usepackage{booktabs}
\usepackage{float}

\begin{document}

    \section{Analysis of Cargo Arrivals for Steel at Frankfurt Station}

    \subsection{Data Source}
    The analysis is based on the provided histogram titled "Distribution for Steel at Frankfurt Station", which visualizes the frequency distribution of steel shipment amounts arriving at this station.

    \begin{figure}[H]
        \centering
        \includegraphics[width=0.9\textwidth]{histogram}
        \caption{Histogram of Steel Amount at Frankfurt Station}
        \label{fig:histogram}
    \end{figure}

    \subsection{Frequency Distribution and Shape}
    The histogram displays the frequency of different steel amounts per arrival.
    \begin{itemize}
        \item The distribution appears unimodal, with the highest frequency occurring in the lowest range of steel amounts (approximately 0-300 units).
        \item The distribution is \textbf{positively skewed (skewed to the right)}. This is indicated by the long tail extending towards higher steel amounts and is confirmed by the mean being greater than the median.
    \end{itemize}

    \subsection{Descriptive Statistics}
    Key descriptive measures derived from the histogram legend:
    \begin{itemize}
        \item \textbf{Measures of Central Location}:
        \begin{itemize}
            \item Mean ($\overline{x}$): $722.10$ units. This average is pulled higher by the larger shipment values in the tail.
            \item Median ($\tilde{x}$): $633.21$ units. This indicates that 50\% of steel shipments were below 633.21 units and 50\% were above. It is less affected by extreme values than the mean.
            \item Mode: The modal class (most frequent) corresponds to the leftmost bar, representing the lowest range of steel amounts (approx. 0-300 units), which has a frequency of 30.
        \end{itemize}
        \item \textbf{Measures of Variability}:
        \begin{itemize}
            \item Minimum: $4.01$ units. The smallest recorded steel amount.
            \item Maximum: $2911.20$ units. The largest recorded steel amount.
            \item Range ($r = \text{Max} - \text{Min}$): $2911.20 - 4.01 = 2907.19$ units. This shows a very large spread between the smallest and largest shipments.
            \item Standard Deviation ($s$): $599.70$ units. This value reflects significant variability in the steel amounts per shipment, indicating that shipment sizes often deviate considerably from the average amount.
        \end{itemize}
    \end{itemize}

    \subsection{Interpretation}
    The analysis of steel shipments arriving at Frankfurt Station reveals:
    \begin{itemize}
        \item Shipments are highly variable in size, ranging from approximately 4 units to over 2900 units.
        \item Most shipments contain relatively smaller amounts of steel (concentrated below the median of 633.21 units), as shown by the high frequency in the lower bins and the positive skewness.
        \item Occasionally, very large shipments arrive (indicated by the tail to the right and the maximum value), which significantly increases the overall average (mean) compared to the median.
        \item The large standard deviation ($s=599.70$) compared to the central tendency measures further emphasizes the lack of consistency in shipment sizes. Understanding the reasons for this variability (e.g., different suppliers, seasonal demand, project types) could be a next step.
    \end{itemize}

\end{document}
