\documentclass{article}
\usepackage[utf8]{inputenc}
\usepackage{geometry}
\usepackage{tikz}
\usepackage{booktabs}
\usepackage{graphicx}
\usepackage{xcolor}
\usepackage{subfig}  % Changed from subfigure to subfig
\geometry{a4paper, margin=2.5cm}

% Only load pgf-pie if you have it installed
\IfFileExists{pgf-pie.sty}{
  \usepackage{pgf-pie}
}{
  % Fallback if pgf-pie is not available
  \usepackage{tikz}
  \newcommand{\pie}[1]{(Pie chart: #1)}
}

\title{Distribuição de Dados por Estação}
\author{US18 - Análise de Dados}
\date{\today}

\begin{document}

\maketitle

\section{Resumo dos Dados}

\begin{table}[h]
\centering
\begin{tabular}{lccc}
\toprule
\textbf{Estação} & \textbf{Comboios} & \textbf{Passageiros} & \textbf{Correio (t)} \\
\midrule
Berlin & 152447 & 54401142 & 14127.36 \\
Frankfurt & 90242 & 53823771 & 2890.04 \\
Hamburg & 244224 & 74362990 & 6807.81 \\
Hannover & 92244 & 41394661 & 1982.04 \\
Stuttgart & 217744 & 68289029 & 2350.31 \\
\bottomrule
\end{tabular}
\caption{Resumo de dados por estação}
\end{table}

\clearpage % Force a page break here

\section{Gráficos de Distribuição}

\begin{figure}[h!] % The ! is important to force it to stay here
\centering
\subfloat[Distribuição de Comboios]{
  \begin{tikzpicture}[scale=0.65]
    \pie{19.1/Berlin (152447), 11.3/Frankfurt (90242), 30.6/Hamburg (244224), 11.6/Hannover (92244), 27.3/Stuttgart (217744)}
  \end{tikzpicture}
}

\vspace{1.5cm} % Vertical space between first and second charts

\subfloat[Distribuição de Passageiros]{
  \begin{tikzpicture}[scale=0.65]
    \pie{18.6/Berlin (54401142), 18.4/Frankfurt (53823771), 25.4/Hamburg (74362990), 14.2/Hannover (41394661), 23.4/Stuttgart (68289029)}
  \end{tikzpicture}
}

\vspace{1.5cm} % Vertical space between second and third charts

\subfloat[Distribuição de Correio]{
  \begin{tikzpicture}[scale=0.65]
    \pie{50.2/Berlin (14127), 10.3/Frankfurt (2890), 24.2/Hamburg (6807), 7.0/Hannover (1982), 8.3/Stuttgart (2350)}
  \end{tikzpicture}
}
\end{figure}

\end{document}